
%%%%%%%%%%%%%%%%%%%%%%%%%%%%%%%%%%%%%%%%%
% HW Template
% LaTeX Template
% Version 1.0 (19/10/18)
% Modified by
% Erdem TUNA
% Halil TEMURTAŞ 
% Enes TAŞTAN 
%%%%%%%%%%%%%%%%%%%%%%%%%%%%%%%%%%%%%%%%%
%
%----------------------------------------------------------------------------------------
%	PACKAGES AND OTHER DOCUMENT CONFIGURATIONS
%----------------------------------------------------------------------------------------
\documentclass[a4paper,12pt]{article}
%-----packages------
\usepackage[a4paper, total={6.5in, 8.5in}]{geometry}
\usepackage[english]{babel}
\usepackage[utf8x]{inputenc}
\usepackage{amsmath}
\usepackage{graphicx}
\usepackage[colorinlistoftodos]{todonotes}
\usepackage{gensymb} % this could be problem
\usepackage{float}
\usepackage{fancyref}
\usepackage{subcaption}
\usepackage[toc,page]{appendix} %appendix package
\usepackage{xcolor}
\usepackage{listings}
\usepackage{xspace}
\usepackage{amssymb}
\usepackage{nicefrac}
\usepackage{gensymb}
\usepackage{fancyhdr}
\usepackage{blindtext}  % for dummy text, use \blindtext or \BlindText
\usepackage[final]{pdfpages}  % pdf include
\usepackage{array} %allows more options in tables
\usepackage{pgfplots,pgf,tikz} %coding plots in latex
\usepackage{capt-of} % allows caption outside the figure environment
\usepackage[export]{adjustbox} %more options for adjusting the images
\usepackage{multicol,multirow,slashbox} % allows tables like table1
%\usepackage[hyperfootnotes=false]{hyperref} % clickable references
\usepackage{epstopdf} % useful when matlab is involved
%\usepackage{placeins} % prevents the text after figure to go above figure with \FloatBarrier 
%\usepackage{listingsutf8,mcode} %import .m or any other code file mcode is for matlab highlighting

%-----end of packages

%-----specifications-----
\definecolor{mGreen}{rgb}{0,0.6,0} % for python
\definecolor{mGray}{rgb}{0.5,0.5,0.5}
\definecolor{mPurple}{rgb}{0.58,0,0.82}
\definecolor{mygreen}{RGB}{28,172,0} % color values Red, Green, Blue for matlab
\definecolor{mylilas}{RGB}{170,55,241}

\setcounter{secnumdepth}{5} % how many sectioning levels to assign numbers to
\setcounter{tocdepth}{5}    % how many sectioning levels to show in ToC

\lstdefinestyle{CStyle}{
	commentstyle=\color{mGreen},
	keywordstyle=\color{magenta},
	numberstyle=\tiny\color{mGray},
	stringstyle=\color{mPurple},
	basicstyle=\footnotesize,
	breakatwhitespace=false,         
	breaklines=true,
	frame=single,
	rulecolor=\color{black!40},                 
	captionpos=b,                    
	keepspaces=true,                 
	numbers=left,                    
	numbersep=5pt,                  
	showspaces=false,                
	showstringspaces=false,
	showtabs=false,                  
	tabsize=2,
	language=C
}

\lstset{language=Matlab,%
	%basicstyle=\color{red},
	breaklines=true,%
	frame=single,
	rulecolor=\color{black!40},
	morekeywords={matlab2tikz},
	keywordstyle=\color{blue},%
	morekeywords=[2]{1}, keywordstyle=[2]{\color{black}},
	identifierstyle=\color{black},%
	stringstyle=\color{mylilas},
	commentstyle=\color{mygreen},%
	showstringspaces=false,%without this there will be a symbol in the places where there is a space
	numbers=left,%
	numberstyle={\tiny \color{black}},% size of the numbers
	numbersep=9pt, % this defines how far the numbers are from the text
	emph=[1]{for,end,break},emphstyle=[1]\color{red}, %some words to emphasise
	%emph=[2]{word1,word2}, emphstyle=[2]{style},    
}


\tikzset{
	desicion/.style={
		diamond,
		draw,
		text width=4em,
		text badly centered,
		inner sep=0pt
	},
	block/.style={
		rectangle,
		draw,
		text width=10em,
		text centered,
		rounded corners
	},
	cloud/.style={
		draw,
		ellipse,
		minimum height=2em
	},
	descr/.style={
		fill=white,
		inner sep=2.5pt
	},
	connector/.style={
		-latex,
		font=\scriptsize
	},
	rectangle connector/.style={
		connector,
		to path={(\tikztostart) -- ++(#1,0pt) \tikztonodes |- (\tikztotarget) },
		pos=0.5
	},
	rectangle connector/.default=-2cm,
	straight connector/.style={
		connector,
		to path=--(\tikztotarget) \tikztonodes
	}
}

\tikzset{
	desicion/.style={
		diamond,
		draw,
		text width=4em,
		text badly centered,
		inner sep=0pt
	},
	block/.style={
		rectangle,
		draw,
		text width=10em,
		text centered,
		rounded corners
	},
	cloud/.style={
		draw,
		ellipse,
		minimum height=2em
	},
	descr/.style={
		fill=white,
		inner sep=2.5pt
	},
	connector/.style={
		-latex,
		font=\scriptsize
	},
	rectangle connector/.style={
		connector,
		to path={(\tikztostart) -- ++(#1,0pt) \tikztonodes |- (\tikztotarget) },
		pos=0.5
	},
	rectangle connector/.default=-2cm,
	straight connector/.style={
		connector,
		to path=--(\tikztotarget) \tikztonodes
	}
}
%-----end of specifications-----


%----commands----
\newcommand\nd{\textsuperscript{nd}\xspace}
\newcommand\rd{\textsuperscript{rd}\xspace}
\newcommand\nth{\textsuperscript{th}\xspace} %\th is taken already
\newcommand{\specialcell}[2][c]{ \begin{tabular}[#1]{@{}c@{}}#2\end{tabular}} % for too long table lines

\newcommand{\blankpage}{
	\- \\[9cm]	
	{ \centering \textit{This page intentionally left blank.} \par }
	\- \\[9cm]
}% For Blank Page

\makeatletter
\renewcommand\paragraph{\@startsection{paragraph}{4}{\z@}%
	{-2.5ex\@plus -1ex \@minus -.25ex}%
	{1.25ex \@plus .25ex}%
	{\normalfont\normalsize\bfseries}}
\makeatother
%-----end of commands-----


\pagestyle{fancy}
\fancyhead[LO,LE]{Halil TEMURTAŞ / 2094522 \\ İclal SATICI / 2094407 }
\fancyhead[RO,RE]{\today}
%\fancyfoot[RO,RE]{\includegraphics[width=2.7cm]{../../../Documents/Logo/logo}}

\begin{document}
\begin{center}
	\textbf{\large EE407 Process Control \\[0.2cm] HW 1} \\
\end{center}

\begin{enumerate}
	\item We will analyse the system shown in the \textit{Figure~\ref{fig:1a}}. 

		\begin{enumerate}
			\item To write the SS model of the system, let us begin with writing fundamental equation describing the system.
			
			$$	F_{Net}~=~m\ddot{x}~=~F-b\dot{x}-kx	$$
			
			Choosing $\underline{x} = {[x~~\dot{x}]}^T$ and $ \underline{y} = [1~~0] $x, we can build our Space-State Model for the system as 
			$$  \underline{\dot{x}}~=~Ax+Bu  ~~ \& ~~  \underline{y}~=~Cx+Du			$$
			
			\[
			\begin{bmatrix}
   				\dot{x} \\
   			 	\ddot{x}
			\end{bmatrix}
			=
			\begin{bmatrix}
    			0 & 1 \\
    			-k/m & -b/m
			\end{bmatrix}			
			\begin{bmatrix}
   				x \\
   			 	\dot{x}
			\end{bmatrix}
			+
			\begin{bmatrix}
   				0 \\
   			 	1/m
			\end{bmatrix}
			u
			\]
			
			where u is the input force F.
			
				\begin{figure}[H]
					\center
					\setlength{\unitlength}{\textwidth} 
					\includegraphics[width=0.4\unitlength]{images/1a}
					\caption{\label{fig:1a} Mass Spring Damper System }
				\end{figure}
			\item 	Simulink Model for the Mass Spring Damper System can be seen at \textit{Figure~\ref{fig:1b}}
				\begin{figure}[H]
					\center
					\setlength{\unitlength}{\textwidth} 
					\includegraphics[width=0.9\unitlength]{images/1b}
					\caption{\label{fig:1b} Simulink Model for the Mass Spring Damper System }
				\end{figure}	\-\\		
			
			\item 	For the following subsections, the simulations are for the model when the applied force is a unit step function starting at t = 5 sec, i.e., $u(t-5)$.
			
%				\begin{figure}[h]
%					\center
%					\setlength{\unitlength}{\textwidth} 
%					\includegraphics[width=0.8\unitlength]{images/1cey}
%					\caption{\label{fig:1c} The System Response for MSD as m = 1kg, b = 0.2Ns/m, k = 1N/m }
%				\end{figure}
			
				\begin{enumerate}
					\item The spring force $F_S~=~kx$ with the direction opposite to the input force $F$ is proportional to the "k" which is the spring force constant . Therefore, when the spring constant decreased, $F_{Net}$ is increased and the displacement $x$ is increased. The figures are consistent with these, \textit{Figure~\ref{fig:1c1}} has small $k$ value and it's final value is bigger than that of \textit{Figure~\ref{fig:1c1a}}.		
			
				\begin{figure}[H]
					\setlength{\unitlength}{\textwidth} 
					\centering
					\begin{subfigure}{.5\textwidth}
  						\centering
  						\includegraphics[width=0.495\unitlength]{images/1cey}
  						\caption{\label{fig:1c1a} The System Response for MSD as m = 1kg, b = 0.2Ns/m, k = 1N/m }
					\end{subfigure}%
					\begin{subfigure}{.5\textwidth}
  						\centering
						\includegraphics[width=0.495\unitlength]{images/1c1ey}
  						\caption{\label{fig:1c1} The System Response for MSD as m = 1kg, b = 0.2Ns/m, k = 0.2N/m}
					\end{subfigure}
					\caption{\label{fig:varyingk} The System Response for MSD with varying spring force constant   }
				\end{figure}
				
%			\begin{figure}[h]
%				\center
%				\setlength{\unitlength}{\textwidth} 
%				\includegraphics[width=0.8\unitlength]{images/1c1ey}
%				\caption{\label{fig:1c1} The System Response for MSD as m = 1kg, b = 0.2Ns/m, k = 0.2N/m }
%			\end{figure}
			
			
			
			\item It is known that $F_{net}=ma=m\ddot{x} $ , then mass and acceleration that is related to position are oppositely proportional, so when mass increased, the rate of change in the velocity of the oscillation is decreased. The \textit{Figure~\ref{fig:1c2a}} and \textit{Figure~\ref{fig:1c2}} are shows the expected effect. While both systems expected to reach same final steady state value, the system in \textit{Figure~\ref{fig:1c2a}} oscillates quickly and reaches the steady state value in shorter time tant the system \textit{Figure~\ref{fig:1c2}}.
			
				\begin{figure}[H]
					\setlength{\unitlength}{\textwidth} 
					\centering
					\begin{subfigure}{.5\textwidth}
  						\centering
  						\includegraphics[width=0.495\unitlength]{images/1cey}
  						\caption{\label{fig:1c2a} The System Response for MSD as m = 1kg, b = 0.2Ns/m, k = 1N/m }
					\end{subfigure}%
					\begin{subfigure}{.5\textwidth}
  						\centering
						\includegraphics[width=0.495\unitlength]{images/1c2ey}
  						\caption{\label{fig:1c2} The System Response for MSD as m = 10kg, b = 0.2Ns/m, k = 1N/m }
					\end{subfigure}
					\caption{\label{fig:varyingm} The System Response for MSD with varying mass   }
				\end{figure}
			
%			\begin{figure}[H]
%				\center
%				\setlength{\unitlength}{\textwidth} 
%				\includegraphics[width=0.8\unitlength]{images/1c2ey}
%				\caption{\label{fig:1c2} The System Response for MSD as m = 10kg, b = 0.2Ns/m, k = 1N/m }
%			\end{figure}
			
			\item  The viscous damping force is proportional to the velocity of the mass, $v=\dot{x}$ with the direction of opposite to the F. In this case, firstly this opposite direction is not so much because of the velocity is small and after some point this velocity value increases and effect the system with more opposite force. Therefore, \textit{Figure~\ref{fig:1c3a}} and \textit{Figure~\ref{fig:1c3}} are expected.
			
			
				\begin{figure}[H]
					\setlength{\unitlength}{\textwidth} 
					\centering
					\begin{subfigure}{.5\textwidth}
  						\centering
  						\includegraphics[width=0.495\unitlength]{images/1cey}
  						\caption{\label{fig:1c3a} The System Response for MSD as m = 1kg, b = 0.2Ns/m, k = 1N/m }
					\end{subfigure}%
					\begin{subfigure}{.5\textwidth}
  						\centering
						\includegraphics[width=0.495\unitlength]{images/1c3ey}
  						\caption{\label{fig:1c3} The System Response for MSD as m = 1kg, b = 2Ns/m, k = 1N/m }
					\end{subfigure}
					\caption{\label{fig:varyingb} The System Response for MSD with varying viscous damping constant   }
				\end{figure}
			
%			\begin{figure}[H]
%				\center
%				\setlength{\unitlength}{\textwidth} 
%				\includegraphics[width=0.8\unitlength]{images/1c3ey}
%				\caption{\label{fig:1c3} The System Response for MSD as m = 1kg, b = 2Ns/m, k = 1N/m }
%			\end{figure}	
			
		\end{enumerate}			
			
			\item 	In this part, the system response of the MSD is observed when an input force are applied for certain time and stopped. We observe that the input force forces system to stabilize its position at non-zero steady-state value. When the input force is removed from system at $t=20~sec$, the steady-state value that the system tries to reach moves back again to zero. The simulink model constructed can be seen at \textit{Figure~\ref{fig:1d1}} and the the system response for this system with input change at $t=20~sec$ can be seen at \textit{Figure~\ref{fig:1d2}}.

				\begin{figure}[H]
					\center
					\setlength{\unitlength}{\textwidth} 
					\includegraphics[width=0.8\unitlength]{images/1d1ey}
					\caption{\label{fig:1d1} Simulink Model for the MSD with Varying Input Force }
				\end{figure}
				
				\begin{figure}[H]
					\center
					\setlength{\unitlength}{\textwidth} 
					\includegraphics[width=0.8\unitlength]{images/1d2ey}
					\caption{\label{fig:1d2} The System Response for MSD as the Input Changes at t=20 s }
				\end{figure}
				
			\item 	Time differences between successive data points seems a varying behaviour in simulation results in step d. This behaviour results with a simulation result that is not as smooth as expected.
				
				
				
			
			
			\item 	As the fixed step is used, the system response can be seen at \textit{Figure~\ref{fig:1f}}. The difference in simulation results with using fixed step option and with using varying step opt,on can be clearly seen at \textit{Figure~\ref{fig:varyfixed}}. The simulation result looks smoother when the fixed step option is used.
			
				\begin{figure}[H]
					\center
					\setlength{\unitlength}{\textwidth} 
					\includegraphics[width=0.8\unitlength]{images/1fey}
					\caption{\label{fig:1f} The System Response for MSD with Desired Parameters in Q1f  }
				\end{figure}
				
				\begin{figure}[H]
					\setlength{\unitlength}{\textwidth} 
					\centering
					\begin{subfigure}{.5\textwidth}
  						\centering
  						\includegraphics[width=0.48\unitlength]{images/1d22ey}
  						\caption{\label{fig:vary} Simulation with Variable Step }
					\end{subfigure}%
					\begin{subfigure}{.5\textwidth}
  						\centering
						\includegraphics[width=0.48\unitlength]{images/1f2ey}
  						\caption{\label{fig:fixed} Simulation with Fixed Step}
					\end{subfigure}
					\caption{\label{fig:varyfixed} Simulation with Variable and Fixed Step   }
				\end{figure}
				
		\end{enumerate}\-\\
	
	
	\item We will analyse the system shown in the \textit{Figure~\ref{fig:2}}.
				\begin{figure}[H]
					\center
					\setlength{\unitlength}{\textwidth} 
					\includegraphics[width=0.4\unitlength]{images/2}
					\caption{\label{fig:2} Propeller Levitated Arm  System }
				\end{figure}	
				
	
	The system can be modelled mathematically as

		$$	mL^2\ddot{\theta}=-c\dot{\theta}-mgLsin(\theta)+u	$$	
	
	where $\theta$ is angular position of the arm, $m$ is the total mass of the propeller and DC motor, $L$ is length of the arm, $c$ is viscous damping coefficient, $g$ is gravitational acceleration and $u$ is thrust produced by the propeller.% and it can be transferred to Laplace domain in order to put it in a block diagram for as 
	
%	$$	mL^2s^2\theta=-cs\theta-mgLsin(\theta)+u	$$

%	assuming that $ \mathcal{L}\{mgLsin(\theta)\}~=~mgLsin(\theta)	$ around certain $\theta$.
	
				
	
		\begin{enumerate}
			\item Simulink Model for the Propeller Levitated Arm can be seen at \textit{Figure~\ref{fig:2a}} and its more compact version can be seen at \textit{Figure~\ref{fig:2a2}}.			
			
				\begin{figure}[H]
					\center
					\setlength{\unitlength}{\textwidth} 
					\includegraphics[width=0.8\unitlength]{images/2a}
					\caption{\label{fig:2a} Simulink Model for the Propeller Levitated Arm }
				\end{figure} 
				
				\begin{figure}[H]
					\center
					\setlength{\unitlength}{\textwidth} 
					\includegraphics[width=0.8\unitlength]{images/2a2}
					\caption{\label{fig:2a2} Simulink Model for the Propeller Levitated Arm using Subsystems}
				\end{figure} 

			
			\-\\
			\item  X, Y and Z represent the followings:
				\begin{itemize}
					\item X represents $\ddot{\theta}$ 
					\item Y represents $\dot{\theta}$ and 
					\item Z represents $\theta$	
				\end{itemize}							
			
			\item This model is valid for all values of the state vector. Number of the state variable and independent row number are equal so the system is consistent with the real dynamics of the system.

			
			%Around $\theta$ values that satisfies
				%$$	\mathcal{L}\{mgLsin(\theta)\}~=~mgLsin(\theta)	$$	
				
			\item Simulation results for the system as $F~=~14Nm$ and as $15Nm$ can be seen at \textit{Figures~\ref{fig:2d2} and \ref{fig:2d1}} respectively.
			
			The slight difference in the thrust results in extremely large change in the system because the system is unstable. When the input force reaches some point, it will not stabilize.
			
%				\begin{figure}[H]
%					\center
%					\setlength{\unitlength}{\textwidth} 
%					\includegraphics[width=0.8\unitlength]{images/2d2}
%					\caption{\label{fig:2d2} The System Response for PLA for m = 1,L =2, g = 9.81, c = 0:5 and input force=14}
%				\end{figure}
				
%				\begin{figure}[H]
%					\center
%					\setlength{\unitlength}{\textwidth} 
%					\includegraphics[width=0.8\unitlength]{images/2d1}
%					\caption{\label{fig:2d1} The System Response for PLA for m = 1,L =2, g = 9.81, c = 0:5 and input force=15  }
%				\end{figure}
				
				
				\begin{figure}[H]
					\setlength{\unitlength}{\textwidth} 
					\centering
					\begin{subfigure}{.5\textwidth}
  						\centering
  						\includegraphics[width=0.495\unitlength]{images/2d2}
  						\caption{\label{fig:2d2} The System Response for PLA for m = 1,\\L =2, g = 9.81, c = 0:5 and input force=14 }
					\end{subfigure}%
					\begin{subfigure}{.5\textwidth}
  						\centering
						\includegraphics[width=0.495\unitlength]{images/2d1}
  						\caption{\label{fig:2d1} \label{fig:2d1} The System Response for PLA for m = 1,\\L =2, g = 9.81, c = 0:5 and input force=15 }
					\end{subfigure}
					\caption{\label{fig:2d12}  The System Response for PLA with varying input   }
				\end{figure}			



		\end{enumerate}
		
		
		
		
		
		
\end{enumerate}
	
	
	
	
\end{document}

%----samples------
%\begin{itemize}
%\item Item
%\item Item
%\end{itemize}

%\begin{figure}[H]
%\center
%\setlength{\unitlength}{\textwidth} 
%\includegraphics[width=0.7\unitlength]{images/logo1}
%\caption{\label{fig:logo}Logo }
%\end{figure}

%\begin{figure}[H]
%	\setlength{\unitlength}{\textwidth} 
%	\centering
%	\begin{subfigure}{.5\textwidth}
%  		\centering
%  		\includegraphics[width=0.48\unitlength]{images/logo1}
%  		\caption{\label{fig:logo1}Logo1 }
%	\end{subfigure}%
%	\begin{subfigure}{.5\textwidth}
%  		\centering
%		\includegraphics[width=0.48\unitlength]{images/logo2}
%  		\caption{\label{fig:logo2}Logo2}
%	\end{subfigure}
%\caption{\label{fig:calisandegree} Small Logos   }
%\end{figure}
	
%\begin{table}[H]
%  \centering
% 
%    \begin{tabular}{c|c|c}
%       $$A$$ & $$B$$ & $$C$$ \\ \hline
%       1 & 2 & 3  \\ \hline
%       2 & 3 & 4  \\ \hline
%       3 & 4 & 5  \\ \hline
%       4 & 5 & 6  
%      
%  \end{tabular}
%  \caption{table}
%  \label{tab:table}
%\end{table}
	
%\begin{table}[H]
%  \centering
% 
%    \begin{tabular}{c|c|c}
%       \backslashbox{$A$}{$a$} & $$\specialcell{ Average deviation \\ after subtracting out the  \\ frequency error }$$ & $$C$$ \\ \hline
%       \multirow{2}{*}{1} & 2 & 3  \\ \cline{2-3}
%        & 3 & 4  \\ \hline
%       3 & \multicolumn{2}{c}{4}  \\ \hline
%       4 & 5 & 6  
%      
%  \end{tabular}
%  \caption{table}
%  \label{tab:table}
%\end{table}
%-----end of samples-----